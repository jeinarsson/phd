\documentclass[thesis.tex]{subfiles}

\begin{document}

In the following Sections I give the context and main results for each of the four projects I am part of.

My main project is the development of an effective equation of motion for a neutrally buoyant spheroid suspended in a simple shear flow, valid when inertial effects are weak but not vanishing. It consists of the appended Papers A-D.

In addition I have taken part in three studies related to the orientational motion of non-spherical particles. They have in common that they involve Jeffery's theory for ellipsoidal particles.
\begin{enumerate}
    \item a microfluidic experiment aiming to validate Jeffery's theory for the rotation of triaxial particles in shear flow (Paper E),
    \item a study of the effects of thermal noise on the orientational dynamics of triaxial particles in shear flow, with implications for the dilute suspension rheology (Paper F),
    \item a description of the rotational modes of disks and rods in isotropic turbulence, combining data from experiments, direct numerical simulations and random flow theory (Paper G).
\end{enumerate}

\chapter{Effective equation of motion for a weakly inertial spheroid in shear flow}

This project is a collaboration with collegues in Cherbourg and Marseille (France), and Stockholm (Sweden). J.R. Angilella (Cherbourg) and F. Candelier (Marseille) have many years of experience in dynamical systems, fluid mechanics and perturbation theory, without which this project would not have landed. T. Ros\'en and F. Lundell in Stockholm are experts in direct numerical simulation of particulate flows by the lattice Boltzmann method, by which we could validate our calculations. I also attribute the initial idea to perform stability analysis on the log-rolling motion under inertial perturbation, however vague at the time, to F. Lundell at a COST meeting in Udine, Italy. 

Paper A is a brief summary of the calculation and result in letter form, while the Papers B \& C contain all details. Paper D describes the direct numerical simulations which validates our theoretical calculation, and shows in detail when the effective equations break down due to finite domain size and increasing importance of inertial effects.

\section{History of problem}

\section{Anatomy of an effective equation}

\begin{align}
    \rho_f \left(\pdiff{}{t}\ve u + (\ve u \cdot \nabla) \ve u\right)=\nabla \cdot \ma \sigma\,.
\end{align}
\begin{align}
    \ma \sigma = -p \ma 1 + 2\mu \ma S
\end{align}
\begin{align}
    \ve u(\ve x, t) &= \dot{\ve y} + \ve \omega \cross (\ve x - \ve y), & \ve x \in S. \nn\\
    \ve u(\ve x, t) &= \ve u^\infty(\ve x, t), & |\ve x-\ve y|\to\infty\,.
\end{align}
\begin{align}
    \ve F &= \int_S \ma \sigma \cdot \rd S\,, \\
    \ve T &= \int_S (\ve x - \ve y) \cross \ma \sigma \cdot \rd S\,.
\end{align}
\begin{align}
    m\ddot{\ve y} = \ve F\,, \quad \diff{}{t}(\ma I \ve \omega) = \ve T\,.
\end{align}

\begin{table}
    \begin{tabular}{ll}
    \toprule
    Symbol & Meaning \\
    \midrule
    $\ve u$      & Fluid velocity       \\
    $p$      & Fluid pressure       \\
    $\ma \sigma$      & Fluid stress tensor       \\
    $\ve y$      & Particle position       \\
    $\dot{\ve y}$      & Particle velocity       \\
    $\ddot{\ve y}$      & Particle acceleration       \\
    $\ve F$      & Force on particle from fluid       \\
    $\ve T$      & Torque on particle from fluid       \\
    $\ve \omega$      & Particle angular velocity       \\
    $\ve u^\infty$      & Undisturbed fluid velocity (far from particle)       \\
    $m$      & Particle mass       \\
    $\ma I$ & Particle moment-of-inertia tensor \\
    $\rho_f$      & Fluid density       \\
    $\rho_p$      & Particle density       \\
    \bottomrule
    \end{tabular}
\end{table}

\section{Results}

\chapter{Measurements of asymmetric rods tumbling in microchannel flow}
Paper E
\section{History of problem}
\section{Results}
\chapter{Effects of thermal noise on the tumbling of triaxial ellipsoids}
Paper F
\section{History of problem}
\section{Results}
\chapter{Rotation of axisymmetric particles in isotropic turbulence}
Paper G
\section{History of problem}
\section{Results}

\end{document}

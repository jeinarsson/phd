\documentclass[thesis.tex]{subfiles}

\begin{document}
\chapter{Linearizing the Navier-Stokes equations}

\section{Stationary reference frame}

Let $\ve u(\ve x, t)$ the flow velocity vector field and $\rho_f$ the density of the fluid. Consider the momentum balance in a volume $\mathcal V$ bounded by surface $\mathcal S$:
\begin{align}
    \pdiff{}{t}\int_\mathcal V  \rho_f \ve u \rd\mathcal V +
    \int_\mathcal S \rho_f\ve u (\ve u \cdot \rd \mathcal S) =
    \int_\mathcal S \ma \sigma \cdot \rd \mathcal S
\end{align}
The first term accounts for changing velocities in the bulk of the volume. The second term accounts for the momentum transfer across the boundary of the volume. The right hand side contains the forces acting on the surface of the volume, $\ma \sigma$ is the stress tensor describing the force per unit area in the fluid. A Newtonian fluid is modeled by $\ma \sigma = -p \ma I + 2\mu \ma S$, where $p$ is the pressure field and $\ma S$ is the symmetric part of the flow field gradient. Body forces are omitted here, but are straightforward to add.

We apply the divergence theorem to the surface integrals and find
\begin{align}
    \pdiff{}{t}\int_\mathcal V  \rho_f \ve u \rd\mathcal V +
    \int_\mathcal V  \left[(\nabla\cdot\rho_f\ve u) \ve u+(\rho_f\ve u\cdot\nabla) \ve u\right] \rd \mathcal V =
    \int_\mathcal V \nabla\cdot\ma \sigma \rd \mathcal V.
\end{align}
This equation holds point-wise, because the volume is arbitrary. In addition we consider an incompressible fluid with constant density, and arrive at the Navier-Stokes equations for an incompressible fluid:
\begin{align}
    \rho_f \left(\pdiff{}{t}\ve u + (\ve u \cdot \nabla) \ve u\right)=\nabla \cdot \ma \sigma\,.
\end{align}
We will now introduce a particle through the no-slip boundary conditions at the particle surface $S$. The particle center-of-mass moves along the trajectory $\ve y(t)$, with angular velocity $\ve \omega$. No-slip means no relative velocity, neither normal nor tangential, between fluid and boundary. Far away from the particle, ``at infinity'', the fluid is not disturbed by the presence of the particle:
\begin{align}
    \ve u(\ve x, t) &= \dot{\ve y} + \ve \omega \cross (\ve x - \ve y), & \ve x \in S. \nn\\
    \ve u(\ve x, t) &= \ve u^\infty(\ve x, t), & |\ve x-\ve y|\to\infty\,.
\end{align}
The forces and torques acting on the particle are in turn determined by integrating the fluid stresses over the particle surface:
\begin{align}
    \ve F &= \int_S \ma \sigma \cdot \rd S\,, \\
    \ve T &= \int_S (\ve x - \ve y) \cross \ma \sigma \cdot \rd S\,.
\end{align}
Finally, the particle trajectory is governed by Newton's equations
\begin{align}
    m\ddot{\ve y} = \ve F\,, \quad \diff{}{t}(\ma I \ve \omega) = \ve T\,.
\end{align}

These equations are non-linear and coupled: convective term, moving boundary condition.

\section{Particle reference frame}

To reason about the non-linear system of equations we change coordinates into a reference frame following the particle. This eliminates the complication of a moving boundary condition in exchange for some non-linear inertial terms in the momentum equation. The new reference frame has a its origin at $\ve y(t)$ and its principal directions are given by a rotation $\ma R$.
If we consider a position vector $\ve x$ its components in lab frame are $x_i$. We denote its components in the particle frame by $x_\alpha$, and the two are related by a translation and a rotation:
\begin{align}
    x_\alpha &= R_{\alpha i}\left(x_i - y_i\right)\,.
\end{align}
A velocity $\ve u = \rd \ve x/\rd t$ measured at $\ve x$ has components $u_i$ in the lab frame, and the components in particle frame are related by
\begin{align}
    u_\alpha &= \diff{R_{\alpha i}}{t}\left(x_i - y_i\right) + R_{\alpha i}\left(u_i - \diff{y_i}{t}\right)\,.
\end{align}
The time derivative of the rotation $\ma R$ is related to the angular velocity $\ve \omega$. More precisely, since $R_{\alpha i}R_{\beta i} = \delta_{\alpha\beta}$, we find after differentiation
\begin{align}
    \diff{R_{\alpha i}}{t} &= -Q_{\alpha \beta} R_{\beta i} = -\lc_{\alpha\gamma\beta}\omega_\gamma R_{\beta i}\,.
\end{align}
Here $\ma Q$ is the anti-symmetric matrix such that $\ma Q\ve x = \ve \omega \cross \ve x$. While perhaps unusual, I use the matrix $\ma Q$ in the following because I find the calculations easier to read than with the cross products. In the final result it is easy to convert to the more familiar notation.

With the angular velocity we find the components of velocity to be
\begin{align}
    u_\alpha &= -Q_{\alpha \beta}R_{\beta i}\left(x_i - y_i\right) + R_{\alpha i}\left(u_i - \diff{y_i}{t}\right) \nn \\
    &= -Q_{\alpha \beta}r_\beta + R_{\alpha i}u_i - R_{\alpha i}\diff{y_i}{t}\,.
\end{align}
Similarly, the components of the acceleration $\partial \ve u/\partial t$ measured at $\ve x$ becomes in particle frame
\begin{align}
    \pdiff{u_\alpha}{t} &= -\diff{Q_{\alpha \beta}}{t}R_{\beta i}\left(x_i - y_i\right) + Q_{\alpha \beta}Q_{\beta \gamma}R_{\gamma i}\left(x_i - y_i\right) \nn\\
    &\quad- 2Q_{\alpha \beta}R_{\beta i}\left(u_i - \diff{y_i}{t}\right) + R_{\alpha i}\left(\pdiff{u_i}{t} - \frac{\rd^2 y_i}{\rd t^2}\right) \nn \\
    &= -\diff{Q_{\alpha \beta}}{t}r_\beta - Q_{\alpha \beta}Q_{\beta \gamma}r_\gamma - 2Q_{\alpha \beta}u_\beta + R_{\alpha i}\pdiff{u_i}{t} -R_{\alpha i}\frac{\rd^2 y_i}{\rd t^2}\,.
\end{align}
Navier-Stokes for the components in lab frame is
\begin{align}
    \rho_f\left(\pdiff{u_i}{t} + u_j \partial_j u_i\right) &= \partial_j \sigma_{ij}\,.
\end{align}
Multiply by $R_{\alpha i}$ and insert
\begin{align}
    \rho_f\left(\pdiff{u_\alpha}{t} + \diff{Q_{\alpha \beta}}{t}r_\beta + Q_{\alpha \beta}Q_{\beta \gamma}r_\gamma + 2Q_{\alpha \beta}u_\beta +R_{\alpha i}\frac{\rd^2 y_i}{\rd t^2}\right.\nn\\
    \left.\quad (R_{\alpha j}u_\alpha + R_{\alpha j}Q_{\alpha \beta}r_\beta + \diff{y_j}{t} ) \partial_j u_\alpha\right) = \partial_j \sigma_{ij}\,.    
\end{align}

\end{document}

%!TEX root=thesis.tex
\documentclass[thesis.tex]{subfiles}

\begin{document}

In the following three Sections I give the context and main results of my recent work. 

In \Secref{effective} I describe the development of an effective equation of motion for the orientation of a neutrally buoyant spheroid suspended in a simple shear flow, valid when inertial effects are weak but not vanishing. In short, we calculate what happens to the Jeffery orbits when the particle Reynolds number is non-zero. The results are contained in the appended Papers A-D.

The other two studies also relate to the orientational motion of non-spherical particles, and their common denominator is that they involve Jeffery's theory for ellipsoidal particles.
The first, in \Secref{experiment}, is a microfluidic experiment aiming to validate Jeffery's theory for the rotation of triaxial particles in shear flow (Paper E). The second, in \Secref{turbulence}, is a description of the rotational modes of small disks and rods in isotropic turbulence, combining data from experiments, direct numerical simulations and random flow theory (Paper F).

\chapter[Effects of inertia]{Effects of inertia on the Jeffery orbits}\label{sec:effective}

This project is a collaboration with collegues in Cherbourg and Marseille (France), and Stockholm (Sweden). J.R. Angilella (Cherbourg) and F. Candelier (Marseille) have many years of experience in dynamical systems, fluid mechanics and perturbation theory, without which this project would not have landed. T. Ros\'en and F. Lundell in Stockholm are experts in direct numerical simulation of particulate flows by the lattice Boltzmann method, by which we could validate our calculations. I also attribute the initial idea to perform stability analysis on the log-rolling motion under inertial perturbation, however vague at the time, to F. Lundell at a COST meeting in Udine, Italy. 

Paper A is a brief summary of the calculation and result in letter form, while the Papers B \& C contain all details. Paper D describes the direct numerical simulations which validates our theoretical calculation, and shows in detail when the effective equations break down due to finite domain size and increasing importance of inertial effects.

\section{History of problem}

The Jeffery equations for an axisymmetric particle in simple shear flow are interesting because their solutions are degenerate, as explained in \Secref{jefferyorbits}. Their solutions form a one-parameter family of periodic orbits. No orbit is preferred over another, so that the initial condition determines the dynamics indefinitely. Jeffery was very aware of this fact, he writes \blockquote{It is obviously undesirable to leave a problem, which is physically quite determinate, in this indeterminate form.}
He further conjectures that \blockquote{this failure is due to the limitations of the theory of the slow motion of a viscous fluid.}
In other words, he believed that inertial corrections were necessary to break the degenaracy. He concludes, referring to inertial corrections, \blockquote[][...]{[...] a more complete investigation would reveal the fact that the particles do tend to adopt special orientations} 
In connection with this discussion Jeffery hypothesized that this preferred orientation would be such that the energy dissipated by viscosity is minimized: a prolate particle ends up log-rolling (long axis along the vorticity), and an oblate particle tumbles (with a diameter along vorticity.)

\citet{saffman1956} made the first attempt to include the non-linear inertial terms. It seems, although details are sparse, that he used an early form of asymptotic matching. For this he acknowledges I. Proudman, who a year later co-authored a paper \cite{proudman1957} on the inertial correction to the drag on a translating sphere, pioneering the use of asymptotic matching in viscous fluid mechanics. But Saffman did not have the proper solution to the outer ``Oseen problem'' for matching, instead he invented a plausible but ad-hoc boundary condition to match the inner expansion. With this method, applied for nearly spherical particles, he found agreement with Jeffery's minimum dissipation hypothesis.

\citet{harper1968} analyzed the rotation of two spheres rigidly constrained by an invisible rod, a so-called dumbbell. In the purely viscous regime a dumbell is equivalent to a prolate spheroid when its aspect ratio approaches infinity. They assumed that both spheres experienced lift forces (calculated by \citet{saffman1965}) independently of each other. By this method they found the opposite of Jeffery's minumum dissipation hypothesis: a slender rod ends up tumbling end-to-end in the flow-shear plane.

More recently \citet{subramanian2005,subramanian2006} re-examined the slender fibre and nearly-spherical limiting cases using a reciprocal theorem \cite{kim1991,lovalenti1993}. Their method is less controversial than those employed by the earlier attemps. However, they found, like \citet{harper1968}, that for small values of the Reynolds number a slender rod tumbles end-to-end in the flow-shear plane. For larger values of the Reynolds number the tumbling orbit is destroyed and replaced by fixed points, which means that the particle stops rotating and aligns, a phenomenen observed in numerical studies \cite{ding2000} (see below). \citet{subramanian2006} found that a nearly spherical prolate particle aligns its long axis along the vorticity, and a nearly spherical oblate particle tumbles, in agreement with \citet{saffman1956}. They remark that the different types of motions for nearly spherical prolate particles, and slender prolate particles \blockquote{suggests a possible bifurcation [...] at an intermediate aspect ratio.}

Meanwhile, several groups began studies of this problem using direct numerical solution of the Navier-Stokes equations, via lattice Boltzmann simulations \cite{feng1995,ding2000,qi2003,yu2007,huang2012,rosen2014,mao2014,rosen2015a,rosen2015b}. These studies reveal a rich structure of dynamical modes for moderate to large values of the Reynolds number. Not only log-rolling or tumbling is possible, but also intermediate limit cycles and alignment with new fixed points. The studies are limited to a few particle shapes, typically $\lambda=1/4$, $\lambda=1/2$, $\lambda=2$ and $\lambda=4$. Instead they focus on the effects of increasing Reynolds numbers, confinement and particle buoyancy.

I enter this chronology sometime in 2013, after submitting a paper \cite{einarsson2014} in which we describe the effects of particle inertia alone, neglecting the fluid inertia. We were intrigued by the results of \citet{subramanian2005,subramanian2006} outlined above, and the fact that no numerical results had shown the predicted log-rolling mode for nearly spherical prolate particles. For example, \citet{qi2003} simulated both oblate ($\lambda=1/2$) and prolate ($\lambda=2$) spheroids, and found for that the oblate particle log-rolls while the prolate particle tumbles, opposite to the existing theoretical prediction. But, as \citet{subramanian2006} points out, there were several possible explanations for this discrepancy. First, the simulations were performed at moderately small Reynolds numbers, but not much smaller than unity, where a perturbation theory should be valid. Second, they were performed in a finite computational domain, whereas the theory is valid for an unbounded shear flow. Third, the particle aspect ratio $\lambda=2$ is not close to unity, whereas the theory assumed $\lambda\approx1$. The parameter ranges where the existing theory should be valid is also where the numerical simulations become computationally impractical. More precisely, small values of the Reynolds number, large distance to the boundaries, and extreme particle shape all add to the computational cost. Therefore any comparisons to existing theory were qualitative.

These facts convinced us to attempt relaxing the assumption of $\lambda\approx1$, and find the exact value of $\lambda$ for the cross-over from log-rolling to tumbling.

During my work several more numerical studies have appeared \cite{rosen2014,mao2014,rosen2015a,rosen2015b}. Despite improved methods and more raw computer power, there was still no evidence of log-rolling prolate spheroids at small values of the Reynolds number. The clearest example is in Fig.~12 of \citet{mao2014}, where they find that the dynamics of a nearly spherical particle ($\lambda=1.2$ and $\lambda=0.8$) also contradicts the theoretical prediction. Their belief is that this \blockquote{may be caused by the influence
of the higher-order effects...}, implying that the value of the Reynolds number in their simulation was out-of-range for the perturbation theory.

\section{Results}

We initially set out to calculate only the linear stability exponents of the log-rolling position. But it turned out that we could compute an explicit correction to Jeffery's equation of motion, which is more useful. Let $\ve n$ be the unit vector along the symmetry axis of the particle, and $\ma O$ and $\ma S$ the antisymmetric and symmetric parts of the shear flow gradient (see \Secref{fluidflows} in Part I for details.) Then the result is
\begin{align}
\eqnlab{ndoteffective}
  \dot {\ve n} &=  
\ma P\left[\ma O \ve n + \Lambda\ma S\ve n\right]\\
&\,+\Reys\ma P\left[b_1 (\ve n \cdot \ma S \ve n)\ma S \ve n
+ b_2 (\ve n \cdot \ma S \ve n)\ma O \ve n 
+b_3\,  \ma O \ma S \ve n 
+ b_4\,  \ma S \ma S \ve n\right]\,. \nn
\end{align}
Here the first row is the result of \citet{jeffery1922}. The projection matrix $\ma P=\ma 1 - \ve n \ve n\transpose$ removes any component of the vector field which is not tangent to the sphere (see also \Secref{jefferyequation}) The scalar parameters $\Lambda$ and $b_\alpha$ depend only on the particle aspect ratio $\lambda$. The shape factor $\Lambda=(\lambda^2-1)/(\lambda^2+1)$ was computed by Jeffery. Our main accomplishment is the calculation of $b_\alpha(\lambda)$.
The result \eqnref{ndoteffective} resolves the inconsistensies between earlier theories, and between theory and numerical simulations. It also conclusively refutes Jeffery's minimum dissipation hypothesis with respect to inertia. I summarise the main conclusions in the following.

\subsection*{Linear stability analysis}

The solution to Jeffery's equation, \Eqnref{ndoteffective} with $\Reys=0$, are the degenerate periodic Jeffery orbits. In terms of the dynamical system \eqnref{ndoteffective} the phase space $S^2$ is covered by a continuous family of marginally stable periodic orbits. But just as Jeffery conjectured, an arbitrarily small amount of inertia breaks this degeneracy. The periodic orbits in phase space are replaced by a set of limit cycles and fixed points. The stable limit cycles and fixed points of \eqnref{ndoteffective} represent the preferred motions of the particle. The form of \Eqnref{ndoteffective} reveals that the vorticity direction $n_i \sim \lc_{ijk}O_{jk}$ is always a fixed point, whatever the values of $\beta_\alpha(\lambda)$. This is fixed point is called \emph{log-rolling} . Similarly, no trajectory of \Eqnref{ndoteffective} can cross the flow-shear plane, and therefore the phase space in that plane must contain either a limit cycle, or a set of fixed points. If it is a limit cycle, the dynamics is called \emph{tumbling}. These general features are due to the symmetries of the shear flow and the axisymmetric particle. 

\subsubsection*{First effects of inertia}
For an arbitrarily small value of $\Reys>0$ the phase space looks almost like the Jeffery orbits, but with a slow \emph{drift} between the orbits. This drift is bounded by the log-rolling and tumbling orbits, and the direction of the drift is determined by the particle shape. We determine the drift by linear stability analysis of \Eqnref{ndoteffective} to order $O(\Reys)$, and find 
\begin{itemize}
    \item $1 < \lambda < \infty$ (prolate): The particle drifts to the stable tumbling limit cycle, whatever the initial condition.
    \item $1/7.3\approx\lambda_c < \lambda < 1$ (thick oblate): The particle drifts to the stable log-rolling fixed point, whatever the initial condition.
    \item $0 < \lambda < \lambda_c$ (thin oblate): Both the log-rolling fixed point and the tumbling limit cycle are stable. Their basins of attraction are separated by one of the intermediate Jeffery orbits which have turned into an unstable limit cycle. The position of the unstable limit cycle depends on the particle shape, shown in Fig.~4 of Paper~C. 
\end{itemize}
Our result \eqnref{ndoteffective} agrees with those of \citet{subramanian2005} in the limit $\lambda\to\infty$ (up to a factor of $8\pi$). However, we find that the earlier results for nearly spherical particles \cite{saffman1956,subramanian2006} are mistaken. We have checked this in two ways. First, we asked F. Candelier to independently analyse the case of nearly spherical particles. We knew that my general solution must match his solution for nearly spherical particles exactly, as $\lambda\to1$. We compared notes and could eventually match our independent calculations. His calculation is described in Paper~B. At that point we compared my solution to \citet{subramanian2005} as $\lambda\to\infty$, and found agreement up to a numerical factor of $8\pi$. We have not identified exactly where in their calculation this factor appears, but the likely culprit is in the definition of Green's functions for Stokes flow (see App.~A.2 in Paper~C.) With these comparisons we were confident enough to submit our calculations for review and publication. The second check of our result is the direct numerical stability analysis by T. Ros\'en and F. Lundell. Our effective equation \eqnref{ndoteffective} agrees very well with the full numerical solution as $\Reys\to0$ and the computational box size becomes large. This comparison is in Paper~D (in particular Fig.~2). We conclude that our effective equation is correct. But we also see that the orientational dynamics of a non-spherical particle in shear flow is sensitive to both confinement (wall-effects), and to higher-order corrections in the shear Reynolds number.

\subsubsection*{Dynamics of oblate particles at finite values of $\Reys$}
In the previous Section I discussed the first effects of inertia, for which we expect the perturbative effective equation \eqnref{ndoteffective} to be valid. For larger values of $\Reys$ we cannot be certain that the dynamics of the effective equation reflects the dynamics of the exact equations. For example, the effective equation can in general not predict at which value of $\Reys$ a disk-shaped particle with $\lambda=1/12$ will cease rotating. Nevertheless, we construct a bifurcation diagram of the effective equation in the parameter space $(\lambda, \Reys)$. We know that any bifurcation line that extends to $\Reys=0$ must \emph{connect} to a corresponding bifurcation line in the parameter space of the exact dynamics. This constrains the possible bifurcation topologies for the exact equations, and may serve as a guide for further numerical analysis.

In the bifurcation diagram for the effective equation three lines extend to $\Reys=0$. Two bifurcation lines describe the value of $\Reys$ where the tumbling orbit ceases to exist and is replaced by a pair of fixed points. Those reach $\Reys=0$ only asymptotically as $\lambda\to0$ and $\lambda\to\infty$, corresponding to infinitely thin disks or rods. The slender-body limit was described by \citet{subramanian2005}. But the third bifurcation line separates the regions of stable and unstable tumbling for oblate particles, referred to in the previous Section. It connects to $\Reys=0$ at $\lambda=\lambda_c\approx1/7.3$. In Paper~D there is numerical evidence that this happens also in the exact equations. This raises the question: where does this bifurcation line go as $\Reys$ increase in the effective equation?
To this end, we compute the bifurcation diagram for oblate particles at small but finite values of $\Reys$. \Figref{phasediagram} is a schematic drawing of this diagram. For larger values of $\Reys$ there are more bifurcations (not shown), but we expect them to be less relevant. The interesting feature of this diagram is the fate of the tumbling orbit bifurcation. It meets several other bifurcation lines in a ``critical point''. 

Numerical simulation of the exact equations reveals many different modes of rotation, depending on parameters such as particle aspect ratio, Reynolds number, confinement ratio and particle buoyancy. One may hope that the effective equation is qualitatively correct in predicting what the first bifurcation is as $\Reys$ increases. The data of \citet{rosen2015b} and Paper~D indicate that the ``critical point'', where several bifurcation lines merge, does exist also in the exact dynamics. However, the bifurcation lines seem to be sensitive to the confinement ratio in the numerical simulations and as of now we do not have enough data to confirm nor refute any claims on equivalence.

\begin{figure}
\begin{overpic}[unit=1mm,width=\linewidth]{figs/bifurcations.pdf}
\put(38,-1){\colorbox{white}{{%
     $0.137$}
}}
\put(56,-1){\colorbox{white}{{%
     $0.245$}
}}
\put(76,-1){\colorbox{white}{{%
     $0.5$}
}}
\put(83,-1){\colorbox{white}{{%
     $0.8$}
}}
\put(92,-1){\colorbox{white}{{%
     $1$}
}}
\put(3,2){\colorbox{white}{{%
     $\lambda$}
}}
\put(95,55){\colorbox{white}{{%
     $\Reys$}
}}
\end{overpic}
\vspace{1em}
\caption{\figlab{phasediagram} Bifurcation diagram for oblate particles in the effective equations.%
\begin{itemize}
    \item For larger $\Reys$ there are more bifurcations not depicted in this figure, for instance an additional pair of fixed points is created in the flow-shear plane.
    \item The Log-rolling fixed point is always stable. It is a spiral below the dashed line and node above.
    \item[A] \begin{itemize}
        \item Tumbling orbit unstable,
        \item No additional fixed points/orbits exist.
    \end{itemize}
    \item[B] \begin{itemize}
        \item Tumbling orbit stable,
        \item Unstable limit cycle separates log rolling FP/tumbling orbit.
    \end{itemize}
    \item[C] \begin{itemize}
        \item Tumbling fixed points exist: Saddle / Stable,
        \item Limit cycle exists.
    \end{itemize}
    \item[D] \begin{itemize}
        \item Tumbling fixed points exist: Unstable / Stable,
        \item Non-trivial saddle point exists in interior near tumbling fixed points.
    \end{itemize}
    \item[E] \begin{itemize}
        \item Tumbling fixed points exist: Unstable / Saddle,
        \item No additional fixed points/orbits exist.
    \end{itemize}
\end{itemize}}
\end{figure}
\chapter[Measurements of asymmetric rods]{Measurements of asymmetric rods tumbling in microchannel flow}\seclab{experiment}

This project is a collaboration with our experimentalist collegues in Gothenburg, Sweden. Roughly, the division of work is that they build and perform the experiment, and we design the specifications and perform the analysis.

The project was initiated with an intention of observing the scattering between Jeffery orbits due to thermal noise. But first we needed to observe plain Jeffery orbits, as a baseline. With hindsight that was naive, given the long list of skilled experimentalists before us who struggled with this: Sec.~II in Paper E is a more or less exhaustive list. 

Our focus has shifted away from thermal noise, to the sensitive dependence on particle shape and initial condition. In \Secref{jefftriaxial} in Part I, I explained how Jeffery's result implies complicated dynamics for triaxial particles, even if the deviation from axisymmetry very small \cite{hinch1979,yarin1997}. We want to observe this effect in experiment.

We previously published two papers \cite{mishra2012,einarsson2013} describing our methods and some initial observations. In Paper~E we describe our most recent measurements. There are two main differences from before. First, we employ an optical trap to arrange particles for the experimental runs. This allows us to use the same particle several times, and to control its initial condition. Second, we have particles made from glass fibres which have a very symmetric cross-section.

In Paper~E we claim to observe both quasi-periodic and chaotic trajectories, for the same particle. Thus we confirm the predictions of \citet{hinch1979} and \citet{yarin1997}. We can claim that the observed trajectories are due to particle shape because of two main reasons. First, we reverse the pressure over the channel at the end of each particle trajectory. The particle must then retrace its trajectory backwards. If they do not, we discard the data. These reversals exclude any non-reversible effects, in particular effects of inertia or thermal noise.

Second, we record several distinct trajectories with different initial conditions a single particle.
\begin{enumerate}
    \item Pressure reversals revert the orientational dynamics, which excludes any non-reversible effects such as inertia or thermal noise,
    \item Several trajectories for the same particle, but with different initial conditions.
\end{enumerate}

\chapter{Rotation rates of particles in turbulence}\seclab{turbulence}
Paper~F started as the synthesis of discussions during a workshop at {\sc nordita} in Stockholm. For those I am grateful especially to E.~Variano and G.~Voth. The paper is a discussion of the rotations of axisymmetric particles in isotropic turbulence. I think the strength of this paper is it's breadth, as it contains pieces of experimental results, numerical results and analytical model calculations.

From what I remember, the discussions started because of confusion between the \emph{rotation rate} and the \emph{tumbling rate} of a particle. In this context rotation rate means magnitude of the angular velocity: $|\ve \omega|$. The tumbling rate is the rate at which the symmetry axis of the particle turns: $|\dot{\ve n}|=|\ve \omega \cross \ve n|$. The two are kinematically related, because
\begin{align}
    |\ve \omega|^2 &= |\dot{\ve n}|^2 + |\ve \omega \cdot \ve n|^2\,.
\end{align}
The difference $|\ve \omega \cdot \ve n|$ is called the \emph{spinning rate}, because it is the rate at which the particle spins around its symmetry axis.

In the paper we make two main observations. First, that the average rotation rate is roughly independent of particle shape. This is true in numerical simulations (Fig.~3 in Paper G), and in experimental measurements (Fig.~5 in Paper G.) This shape independence is unexpected, in particular because the average tumbling rate has a strong shape dependence \cite{parsa2012,gustavsson2014}. Therefore it turns out that the shape dependence of the average spinning rate almost exactly cancels the shape dependence of the average tumbling rate, as to make the total rotation rate shape independent. The reasons for this cancellation are still not known. However, in the paper we show that the average rotation rate of a particle in a random flow field is \emph{not} shape independent. This implies that the cancellation is due to the properties of the turbulent flow, and not inherent in the equations of motion.

The second main observation is on the \emph{instantaneous} rotation rates of particles. Although the average rotation rates for a thin disk and a slender rod are almost the same, their trajectories are qualitatively very different.

A key feature of turbulence is the existence of \emph{vortex tubes} \cite{she1990}. They are regions of strong vorticity, created by stretching of a large vortex into a thinner but more intensive vortex. These regions typically are long-lived, compared to the average rate of change in the flow. In these vortex tubes, rods tend to rotate such that it keeps aligned with the direction of the vorticity. The vorticity makes them spin around their own symmetry axis. But disks instead align a diameter with the vorticity, and the vorticity makes them tumble. But as the disk tumbles, the tumbling rate alternates between being faster and slower than vorticity, because of the flow strain. An example of this is shown in the first panel of Fig.~1 in Paper G. The rotation rate of the rod varies smoothly, and is very close to the strength of the vorticity. The rotation rate of the disk oscillates strongly, but is on average close to the strength of the vorticity.

We may partly understand these observations by a very simple picture. The effect of a rotational flow $\ve u_R= \ve \Omega \cross \ve r$ is to rotate a particle around $\ve \Omega$. The effect of a strain flow $\ve u_S=\ma S \ve r$ is to align a long axis of a particle with the strongest eigendirection of $\ma S$. The simple picture is that the same strain that stretches and intensifies a vortex to a vortex tube, will also align the axes of any nearby particles with $\ve \Omega$. Therefore long axes of rods, and diameters of disks tend to align with $\ve \Omega$ in these regions. With this alignment it follows that rods spin and disks tumble because of the strong vorticity.

This simple argument cannot explain why the rotation rate of the disk happens to average to the same value as that of the rod. The details of the tumbling rate depends on how the vorticity $\ve \Omega$, and the particle direction $\ve n$, are aligned relative to the eigensystem of $\ma S$. The details and implications of these alignments are important open questions. In the random-flow model these alignments are very weak, and that is the underlying reason for the shape-dependence of the average rotation rate.

\chapter{Conclusions}

\section{Summary}

\section{Outlook}

\subsubsection{Extensions to Papers~A-D}

There are two obvious extensions to our work on the orientational motion of spheroids in simple shear. First, the effect of confinement of the particle by nearby walls is substantial. The nearby boundaries affect the nature of the inertial correction. Likely this effect can be computed by a method of images. The second extension is to compute one more order of perturbation in $\Reys$. From the numerical results in Paper~D it is clear that the next order makes a substantial difference, and that it should allow for a strong connection from the analytical results to the lattice Boltzmann simulations. In the case of spherical particles, the lowest order correction is $O(\Reys^{3/2})$.

\subsubsection{Singular perturbation theory}

The reason that the calculation described in Papers~A-C is conceptually straightforward is that the Stokes flow works as the zeroth order flow field in the reciprocal theorem integral. Perturbation theory for small values of the Reynolds number is infamous, because the Stokes flow field is not a uniformly valid approximation of the flow field as $\Reys\to0$. In our case this did not matter, as the erronous contribution to the volume integral is small. But in cases involving translational motion the volume integral diverges. We may not, for example, reproduce the Saffman lift force on a sphere translating in simple shear with just the Stokes flow field and the reciprocal theorem.

In order to solve most problems, it is necessary to construct uniformly valid flow fields to lowest order. This usually requires singular perturbation theory, of which asymptotic matching is perhaps the most common technique in fluid dynamics. I believe that understanding these methods is a necessary next step.


\end{document}
